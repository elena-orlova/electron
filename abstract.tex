\documentclass[a4paper, 12pt]{article}
\usepackage[14pt]{extsizes}

\makeatletter
\newenvironment{sqcases}{%
  \matrix@check\sqcases\env@sqcases
}{%
  \endarray\right.%
}
\def\env@sqcases{%
  \let\@ifnextchar\new@ifnextchar
  \left\lbrack
  \def\arraystretch{1.2}%
  \array{@{}l@{\quad}l@{}}%
}
\makeatother
\usepackage{float}
\usepackage{caption}
\usepackage{subcaption}
\usepackage{xcolor}

\usepackage[top=2cm, bottom=2cm, left=2cm, right=1cm]{geometry}
%\renewcommand{\baselinestretch}{1.2}
%\newcommand{\jj}{\righthyphenmin=20 \justifying}
\usepackage{ragged2e}
\justifying
\usepackage{booktabs}
\usepackage[english,russian]{babel}
\usepackage{amssymb}
\usepackage{enumerate}
\usepackage{units}
\long\def\comment{}

\newtheorem{theorem}{Theorem}
\newtheorem{lemma}{Lemma}
\usepackage{listings} 
\lstset{language=Python} 


\usepackage{tikz}
\usepackage{tikz-3dplot}

\usepackage{caption}
\usepackage{subcaption}
\usepackage{graphicx}

\graphicspath{{./pictures/}}
\DeclareGraphicsExtensions{.pdf,.png,.jpg,.svg}

\usepackage{pdfpages}
\usepackage{amsmath}
\usepackage{textcomp}
\usepackage{gensymb}
\usepackage{physics}
\usepackage{hyperref}


\newcommand*\diff{\mathop{}\!\mathrm{d}}
\newcommand{\slfrac}[2]{\left.#1/#2\right.}
\newcommand{\Int}[4]{\displaystyle \int_{#1}^{#2} #3 \diff #4}
\newcommand{\Intu}[3]{\displaystyle \int_{#1}^{#2} #3}

\addto\captionsrussian{\renewcommand{\figurename}{Fig.}}
\addto\captionsrussian{\renewcommand{\refname}{References}}
\addto\captionsrussian{\renewcommand{\contentsname}{Contents}}
\renewcommand{\figurename}{Fig.}
\renewcommand{\refname}{References}
\renewcommand*\contentsname{Contents}

\pdfsuppresswarningpagegroup=1

\begin{document}
\textbf{Abstract}\\

One of the most actual problems in atomic and nuclear physics is calculation of the energy spectrum of a particle in a spherically symmetric potential, i.e. potential that depends only on the distance between the particle and the force center.

During the working on this task various numerical methods were implemented to calculate the particle spectrum in spherically symmetric potentials. Specifically the following potentials were used: potential of the infinite and the finite spherical wells; potential of the isotropic harmonic oscillator; the Coulomb potential; potentials of Woods-Saxon, Hulthen, Morse and Yukawa. The spherically symmetric potential of the strong neutron-proton interaction in the deuteron model was also investigated.

\textbf{Аннотация}

Одной из актуальных задач в атомной и ядерной физике является расчет энергетического спектра частицы в сферически симметричном потенциале, который зависит только от расстояния между частицей и силовым центром.

В ходе выполнения настоящей курсовой работы были реализованы различные численные методы для расчета спектра частиц в сферически симметричных потенциалах. Конкретно были использованы такие потенциалы как: потенциал бесконечной и конечной сферической потенциальной ямы; потенциал изотропного гармонического осциллятора; кулоновский потенциал; потенциалы Вуда-Саксона, Хюльтена, Морса и Юкавы. Также исследовался сферически симметричный потенциал сильного взаимодействия нейтрона и протона в модели дейтрона.
\end{document}