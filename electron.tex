\documentclass[a4paper, 12pt]{article}

\makeatletter
\newenvironment{sqcases}{%
  \matrix@check\sqcases\env@sqcases
}{%
  \endarray\right.%
}
\def\env@sqcases{%
  \let\@ifnextchar\new@ifnextchar
  \left\lbrack
  \def\arraystretch{1.2}%
  \array{@{}l@{\quad}l@{}}%
}
\makeatother
\usepackage{float}
\usepackage{caption}
\usepackage{subcaption}
\usepackage{xcolor}

\usepackage[top=2cm, bottom=2cm, left=2cm, right=1cm]{geometry}
%\renewcommand{\baselinestretch}{1.2}
%\newcommand{\jj}{\righthyphenmin=20 \justifying}
\usepackage{ragged2e}
\justifying
\usepackage{booktabs}
\usepackage[russian]{babel}
\usepackage{amssymb}
\usepackage{enumerate}
\usepackage{units}

\usepackage{color}   %May be necessary if you want to color links
\usepackage{hyperref}
\hypersetup{
    colorlinks=true, %set true if you want colored links
    linktoc=all,     %set to all if you want both sections and subsections linked
    linkcolor=blue  %choose some color if you want links to stand out
}

\long\def\comment{}

\newtheorem{theorem}{Theorem}
\newtheorem{lemma}{Lemma}

\usepackage{tikz}
\usepackage{tikz-3dplot}

\usepackage{caption}
\usepackage{subcaption}
\usepackage{graphicx}

\graphicspath{{./pictures/}}
\DeclareGraphicsExtensions{.pdf,.png,.jpg,.svg}

\usepackage{pdfpages}
\usepackage{amsmath}
\usepackage{textcomp}
\usepackage{gensymb}
\usepackage{physics}

\newcommand*\diff{\mathop{}\!\mathrm{d}}
\newcommand{\slfrac}[2]{\left.#1/#2\right.}
\newcommand{\Int}[4]{\displaystyle \int_{#1}^{#2} #3 \diff #4}
\newcommand{\Intu}[3]{\displaystyle \int_{#1}^{#2} #3}

\addto\captionsrussian{\renewcommand{\figurename}{Fig.}}
\addto\captionsrussian{\renewcommand{\refname}{References}}
\renewcommand{\figurename}{Fig.}
\renewcommand{\refname}{References}

\addto\captionsrussian{\renewcommand{\contentsname}{Table of Contents}}
\pdfsuppresswarningpagegroup=1

\begin{document}
	%-------------------------------------------------TITLE----------------------------------------------------------
	\begin{center}
		\large
		\textbf{National Research University}			
		\\
		\textbf{Higher School of Economics}									\\[3 cm]
		\textbf{Moscow Institute of Electronics and Mathematics}			\\[2 cm]
									
	\end{center}
	
	\begin{center}
		\large
		\textbf{Department of Applied Mathematics}		\\[3 cm]
	\end{center}
	
	
	
	\begin{center}
		\large
		\textbf{Calculation of Electron Spectrum in Spherically Symmetric Potentials}		\\[6 cm]
		
	\end{center}
	
	\begin{flushright}
		\large
		\textbf{Orlova Elena}				\\
		\textbf{Scientific Supervisor:}			\\
		\textbf{Ikhsanov Renat}		\\
	\end{flushright}
	
	\vfill
	\begin{center}
		\large
		\textbf{Moscow 2017}
	\end{center}
	\pagebreak
	
	
	
    %-------------------------------------------------------------1-----------------------------------------------------------
	

\tableofcontents
\pagebreak
	
\section{Introduction}

An important problem in quantum mechanics is a behaviour of a particle in a spherically symmetric potential, which depends only on the distance between the particle and a defined center point. An electron in the potential derived from Coulomb's law is well-known example. The problem can be used to describe a hydrogen-like atom. 


The spectrum of particle consists of all possible values of energy. The study of particle spectra allows us to see the global picture of a particle behavior. 


We are interested in finding a spectrum of an electron in spherically symmetric potentials, such as: the infinite and finite spherical potential well, the hydrogen-like atom; the Woods-Saxon, Hulthen, Morse, Yukawa potentials and the deutron model. As a main source of analytical results we use \cite{flugge2012practical}.

\section{Theoretical part}
\subsection{Differential operators in spherical coordinates}\label{dif_op_3d}


Suppose we have standard three dimension axes . A spherical coordinate system is a coordinate system where the position of a point is specified by three quantities: the radial distance $r$ of that point from the origin, its polar angle $\theta$ measured between polar axis (positive $z$-axis) and the line from the origin to the point, and the azimuth angle $\phi$ between the positive $x$-axis  and the line's from the origin to the point orthogonal projection on a reference plane $Oxy$.


\tdplotsetmaincoords{60}{110}
%
\pgfmathsetmacro{\rvec}{.8}
\pgfmathsetmacro{\thetavec}{30}
\pgfmathsetmacro{\phivec}{60}
%
\begin{figure}[h!]
\begin{center}
\begin{tikzpicture}[scale=5,tdplot_main_coords]
    \coordinate (O) at (0,0,0);
    \draw[thick,->] (0,0,0) -- (1,0,0) node[anchor=north east]{$x$};
    \draw[thick,->] (0,0,0) -- (0,1,0) node[anchor=north west]{$y$};
\draw[thick,->] (0,0,0) -- (0,0,1) node[anchor=south]{$z$};
    \tdplotsetcoord{P}{\rvec}{\thetavec}{\phivec}
    \draw[-stealth,color=red] (O) -- (P) node[above right] {$P$};
    \draw[dashed, color=red] (O) -- (Pxy);
    \draw[dashed, color=red] (P) -- (Pxy);
    \tdplotdrawarc{(O)}{0.2}{0}{\phivec}{anchor=north}{$\phi$}
    \tdplotsetthetaplanecoords{\phivec}
    \tdplotdrawarc[tdplot_rotated_coords]{(0,0,0)}{0.5}{0}%
        {\thetavec}{anchor=south west}{$\theta$}
\end{tikzpicture}
\end{center}
\caption{Spherical coordinates}
\end{figure}
Cartesian coordinates to Spherical coordinates:
\begin{equation} \label{sph_coord}
\begin{cases}
	{x}= {r} \sin{\theta}\cos{\phi} \\
	{y} = {r} \sin{\theta}\sin{\phi} \\
	{z} = {r}\cos{\theta}
\end{cases}
\end{equation}
Laplacian is a differential operator given by the divergence of the gradient of a function on Euclidean space. Laplace operator in Cartesian coordinates:
 	$$\nabla^2= \frac{\partial^2}{\partial x^2}  + \frac{\partial^2}{\partial y^2} + \frac{\partial^2}{\partial z^2}$$
So, using \eqref{sph_coord} we can find Laplacian in spherical coordinates:
\begin{equation}\label{sph_lap}
	\nabla^2 = \frac{1}{r^2} \frac{\partial}{\partial r}({r^2}\frac{\partial f}{\partial r})+\frac{1}{r^2\sin{\theta}}\frac{\partial}{\partial \theta}(\sin{\theta}\frac{\partial f}{\partial \theta})+\frac{1}{r^2\sin^2{\theta}}(\frac{\partial^2 f}{\partial \phi^2})
\end{equation}
Spherical coordinates are specially convenient in case when particle potential depends only on distance from some point.


\subsection{Some  necessary special functions}
For working with Laplasian in spherical coordinates several special functions often appear. We will describe them in two following paragraphs.
\subsubsection{Spherical functions}
Spherical harmonics are the angular part of the family of orthogonal solutions of the Laplace equation written in spherical coordinates \eqref{sph_lap}:
$$\nabla^2 {f} = \frac{1}{r^2} \frac{\partial}{\partial r}({r^2}\frac{\partial f}{\partial r})+\frac{1}{r^2\sin{\theta}}\frac{\partial}{\partial \theta}(\sin{\theta}\frac{\partial f}{\partial \theta})+\frac{1}{r^2\sin^2{\theta}}(\frac{\partial^2 f}{\partial \phi^2}) = 0$$
Suppose we can find solutions in the form $$f({r}, \theta, \phi) = R(r)Y(\theta, \phi)$$
Separation of variables gives two differential equations:
$$\frac{1}{R}\frac{\partial}{\partial r}(r^2 \frac {dR}{dr}) = \lambda$$
\begin{equation}\label{eq2}
	\frac{1}{Y}\frac{1}{\sin{\theta}}\frac{\partial}{\partial \theta}(\sin{\theta}\frac{\partial Y}{\partial \theta})+\frac{1}{Y}\frac{1}{\sin^2{\theta}}\frac{\partial^2 Y}{\partial \phi^2} = -\lambda
\end{equation}
The solution of \eqref{eq2} is represented in following form, called spherical garmonics of degree $l$ and order $m$:
\begin{equation}\label{sph_garm}
	Y_l^m(\theta, \phi)=\varepsilon \sqrt{\frac{(2l+1)}{4\pi}\frac{(l-|m|)!}{(l+|m|)!}}e^{im\phi}P_l^m(\cos{\theta}),
\end{equation}
 where $\varepsilon = (-1)^m$ for $m\geq0$ and $\varepsilon=1$ for $m \leq 0$, $-l \leq m \leq l$ and $P_l^m $ is associated with Legendre Function, defined by:
	$$P_l^m(x)=(1-x^2)^{\frac{|m|}{2}}(\frac{d}{dx})^{|m|}P_l(x),$$
where $P_l(x) $ is Rodrigues formula:
	$$P_l(x) = \frac{1}{2^l l!}(\frac{d}{dx})^l (x^2-1)^l.$$
Notice, that $m$ and $l$ are integers. For any given $l$ there are $(2l+1)$ possible values for $m$: $l=0, 1...$, $m = -l, -l+1,..., -1, 0, 1, .., l-1, l.$\\
The general solution to Laplace's equation is linear combination of the spherical harmonic functions multiplied by the appropriate scale factor $r^l$:
	$$f(r, \theta. \phi) = \sum_{l=0} ^{\infty}\sum_{m=-l} ^{l}f_l^mr^l Y_l^m(\theta, \phi).$$
where the $f_l^m$ are constants.


\subsubsection{Bessel functions}
Bessel functions are canonical solutions $y(x)$ of following differential equation:
\begin{equation}\label{bess_eq}
x^2\frac{d^2 y}{dx^2}+x\frac{dy}{dx}+(x^2-\alpha^2)y = 0,
\end{equation}
where $\alpha$ is an arbitrary complex number, called the order of the Bessel function.\\
The most important cases are for $\alpha$ an integer or half-integer. Spherical Bessel functions with half-integer $\alpha$ are obtained when the Helmholtz equation is solved in spherical coordinates:
$$\nabla^2 A+k^2 A=0,$$
where $k$  is a wave number and $A$ is an amplitude.\\
When solving the Helmholtz equation in spherical coordinates by separation of variables, the radial equation has the form:
$$x^2\frac{d^2 y}{dx^2}+2x\frac{dy}{dx}+[x^2-n(n+1)]y = 0$$
The two linearly independent solutions to this equation are called the spherical Bessel functions $j_n$ and $y_n$:
\begin{equation}\label{bess_func1}
	j_n(x) = (-x)^n\bigg(\frac{1}{x} \frac{d}{dx}\bigg)^n \frac{\sin{x}}{x}
\end{equation}
\begin{equation}\label{bess_func2}
	n_n(x) = (-x)^n\bigg(\frac{1}{x} \frac{d}{dx}\bigg)^n \frac{\cos{x}}{x}
\end{equation}
\begin{figure}[h!]
\centering
\begin{subfigure}{.5\textwidth}
  \centering
  \includegraphics[width=1.0\linewidth]{bessel1.pdf}
  \caption{$j_n(x)$}
  \label{fig1:bessel1}
\end{subfigure}%
\begin{subfigure}{.5\textwidth}
  \centering
  \includegraphics[width=1.0\linewidth]{bessel2.pdf}
  \caption{$y_n(x)$}
  \label{fig:bessel2}
\end{subfigure}
\caption{Spherical Bessel functions}
\label{fig:bessel}
\end{figure}


\subsection{Angular momentum}
The angular momentum $L$ of a material point relative to a reference point is determined by the cross product of particle's position vector $r$(relative to some origin) and its momentum vector $p$:
$$\mathbf{L = r \times  p},$$
which is to say,
%$$L_x=y p_z-z p_y, L_y=z p_x-x p_z, L_z=x p_y-y p_x$$%
$$\boldsymbol{L} = \begin{vmatrix} \boldsymbol{i} & \boldsymbol{j} & \boldsymbol{k} \\ 
x & y & z \\
p_x & p_y & p_z
\end{vmatrix}$$
This definition can be carried over to quantum mechanics, by reinterpreting $r$ as the quantum position operator   $\hat x=x$  and $p$ as the quantum momentum operator $\hat p=\frac{\hbar}{i}\nabla$:

$$\boldsymbol{L} = \frac{\hbar}{i}\begin{vmatrix} \boldsymbol{i} & \boldsymbol{j} & \boldsymbol{k} \\ 
x & y & z \\
\frac{\partial}{\partial x} & \frac{\partial}{\partial y} & \frac{\partial}{\partial z}
\end{vmatrix}$$
Writing this in coordinates, we obtain:

$$L_x = \frac{\hbar}{i}(y \frac{\partial}{\partial z}-z \frac{\partial}{\partial y}), ~L_y = \frac{\hbar}{i}(z \frac{\partial}{\partial x}-x \frac{\partial}{\partial z}), ~L_z = \frac{\hbar}{i}(x \frac{\partial}{\partial y}-y \frac{\partial}{\partial x})$$
In the next section we will consider the eigenvalues and eigenfunctions of these operators.

\subsubsection{Eigenvalues and eigenfunctions}
%Matrix multiplication is not, in general case, commutative; the difference between the two orderings is called commutator:
%$$[S, T] = ST-TS$$
%In a complex vector space, every linear transformation has special vectors, which are transformed into simple multiples of themselves, 
%$$\hat T \ket{\alpha} = \lambda \ket{\alpha},$$
%$\bra{\Psi}\ket{\Psi}$ $\expval{A}{\Psi}$
%they are called eigenvecrors of the transformation, ans the number $\lambda$ is their eigenvalue. \\[0.5cm]
%n function spaces operators, such as $d/dx, d^2/dx^2$ or $x$ behave as linear transformations. So, the eigenvectors of an operator $\hat T$ are called eigenfunctions:
%$$\hat T f(x)=\lambda f(x)$$
%$L_x$ and $L_y$ do not commute; if we test their action to $f(x, y, z)$, we notice, that:
%$$[L_x, L_y]f = (\frac{\hbar}{i})^2 (y \frac{\partial}{\partial x} -x \frac{\partial}{\partial y})f = i \hbar L_z f,$$
%so
%\begin{equation}\label{comm_L}
%	[L_x, L_y] = i \hbar L_z 
%\end{equation}
%It follows also that 
%$$[L_y, L_z] = i \hbar L_x $$
%$$[L_z, L_x] = i \hbar L_y$$
We have the angular momentum operator $\hat L = (\hat L_x,\hat L_y, \hat L_z),$ and these components sutisfy the following commutation relations:
\begin{equation}\label{comm_L}
	[\hat L_i, \hat L_j] =i \hbar \epsilon_{ijk}  \hat L_k, ~~~i, j, k =1,2,3
\end{equation}
where $\epsilon_{ijk}$ is Levi-Civita symbol. The definition in three dimension:  $\epsilon_{ijk}$ is $1$ if $(i, j, k)$ is an even permutation of $(1, 2, 3)$, $−1$ if it is an odd permutation, and $0$ if any index is repeated.\\
And
$$[\hat L_i, \boldsymbol{\hat L^2}] = 0, $$
where $\mathbf{\hat L^2 = \hat L_x^2+\hat L_y^2+\hat L_z^2} .$\\
The angular momentum operators are commonly used in the solution of spherical symmetry problems. Then the momentum in the spherical coordinates: 
$$-\frac{1}{\hbar^2}L^2=\frac{1}{\sin \theta}\frac{\partial}{\partial \theta}(\sin \theta \frac{\partial}{\partial \theta})+\frac{1}{\sin^2 \theta}\frac{\partial^2}{\partial \phi^2}.$$
The eigenvalues and eigenfunctions are following:
$$L^2 f_l^m = \hbar^2l(l+1) f_l^m,$$
$$L_z f_l^m = \hbar m f_l^m,$$
where $f_l^m = Y_l^m$ - spherical harmonics, defined in \eqref{sph_garm}.




%Using these fundamental commutation relations the entire theory of angular momentum can be deduced.
%But the square of the total angular momentum,
%$$L^2 = L_x^2+L_y^2+L_z^2$$
%oes commutate with $L_x$ :
%$$[L^2, L_x] = 0$$
%It follows:
%$$[L^2, L_y] = 0$$
%$$ [L^2, L_z] = 0, $$
%or, 
%$$[L^2, L] = 0.$$



\subsection{Spin}
In classical mechanics, an object's rotation around a fixed axis  is described by two kinds of angular momentum: orbital ($L = r \times p$), associated with the motion of the center of mass, and spin($S = I \omega$), associated with motion about the center of mass. An analogous things take place in quantum mechanics: in addition to orbital angular momentum, associated with the motion of electron around the nucleus (in case of hydrogen ), and which  described by spherical harmonics, but electron also carries another form of angular momentum, which has nothing to do with motion in space, but which is somewhat analogous to classical spin. Notice, that electron is a structureless point particle, and its spin angular momentum cannot be decomposed into orbital one of constituent parts. So, elementary particles have intrinsic angular momentum $S$ and "extrinsic" angular momentum $L.$\\
The existence of spin angular momentum was proven in experiments, for instance the Stern-Gerlach experiment.\\
The algebraic  theory of spin is the same as theory for orbital angular momentum. So, there are fundamental commutation relations as \eqref{comm_L}.
%\begin{equation}\label{comm_S}
%[S_x, S_y] = i \hbar S_z, ~~~~[S_y, S_z] = i \hbar S_x, ~~~~ [S_z, S_x] = i \hbar S_y.
%\end{equation}
%It follows that the eigenvectors satisfy
%$$S^2 \ket{s ~m} = \hbar^2 s(s+1) \ket{s~m},$$
%$$S_z \ket{s ~m} = \hbar m \ket{s~m}.$$
But these time the eigenvectiors are not spherical harmonics (they are not functions of $\theta$ and $\phi$), so there is no reason to exclude half-integers values of $s$ and $m$:
$$s = 0, \frac{1}{2}, 1, ...;  ~ m = -s, -s+1,...,s-1, s. $$

\subsubsection{Spin $\frac{1}{2}$}
Important case is $s = \frac{1}{2}$, corresponds to, for example,  electron. There are two eigenstates: $\displaystyle{\ket{\frac{1}{2} ~ \frac{1}{2}}}$ and $\displaystyle{\ket{\frac{1}{2} ~ -\frac{1}{2}}}
$. %Using these as basis vectors, the general state a particle with the spin $\frac{1}{2}$ can be described as a two-element column matrix, called spinor:
%$$\chi = \begin{pmatrix} a \\ b  \end{pmatrix} = a \chi_+ + b \chi_-,$$ with $\chi_+ = \begin{pmatrix} 1\\ 0  \end{pmatrix}, ~~ \chi_- = \begin{pmatrix} 0\\ 1  \end{pmatrix}.$\\
Let's add the following discrete variable: spin quantum number $\sigma.$ Then the particle's wave function becomes:
$\Psi = \Psi(x,y,z, \sigma, t), $ and normalization condition becomes $\displaystyle{\sum_i \iint |\Psi_{\sigma_i} (x,y,z, \sigma, t)|^2dV=1}.$ \\
For electrons with $s = \frac{1}{2}$: $\sigma_1 = \frac{1}{2}$ and $\sigma_2 = -\frac{1}{2}$. Then  the spin wave function is:
$$\displaystyle{\phi(t) = 
\begin{pmatrix}
	\chi_{1/2}(t) \\
	\chi_{-1/2}(t)
\end{pmatrix} = 
\begin{pmatrix}
	\chi_1 (t) \\
	\chi_2(t)
\end{pmatrix}} - \text{spinor}.$$
The normalisation condition is:
$$|\chi_1| ^2 + |\chi_2|^2 =1.$$
Hermitian inner product of spinors is following:
$$\phi_1 = \begin{pmatrix}
	a_1 \\
	a_2
\end{pmatrix}, ~ \phi_2 = \begin{pmatrix}
	b_1 \\
	b_2
\end{pmatrix} \Rightarrow \bra{\phi_1}\ket{\phi_2} = a_1^* b_1 + a_2^* b_2.$$
$|a_1|^2$ is a probability of having the particle with the spin  $\frac{1}{2}$
The quantum mechanical operators associated with this spin observables are:
$$S = \frac{\hbar}{2}\sigma,$$\\
For  particles with spin $\frac{1}{2}$  $\sigma_x$, $\sigma_y$, $\sigma_z$ are three matrices, called  Pauli matrices, given by:
$$\sigma_x = 
\begin{pmatrix}
0 & 1 \\
1 & 0
\end{pmatrix} ~~~
\sigma_y = 
\begin{pmatrix}
0 & -i \\
i & 0
\end{pmatrix} ~~~
\sigma_z = 
\begin{pmatrix}
1 & 0\\
0 & -1
\end{pmatrix}$$
$S_x, ~S_y$ and $S_z$ satisfy  canonical momentum commutation relations. It is easy to check, that Pauli matrices are hermitian and unitary.



\subsection{Perturbation theory}
Perturbation theory is the method for the approximate solution of problems in theoretical physics in case when the problem has a small parameter, and for the problem without this parameter it is known an exact solution. Solutions have a form:
$$A = A^{(0)}+\epsilon A^{(1)}+ \epsilon^2 A^{(2)}+\hdots,$$
where $A^{(0)}$ is solution of unperturbed problem, $\epsilon$ is a small parameter, the coefficients $A^{(n)}$ are found by successive approximations.\\
So, we can write Hamiltonian in form:
$$H = H^{(0)} +V,$$
where $H^{(0)}$ is unperturbed Hamiltonian for which a solution is known, $V$ is perturbation.\\
The problem is to find the eigenfunctions of the Hamiltonian (stationary states), and the corresponding energy levels. For this purpose Feynman-Hellman lemma is used, which claims:

\begin{lemma}
Suppose a Hamiltonian $H(\lambda)$, $\lambda$ - parameter, and some normalized energy eigenstate $\ket{\psi_n(\lambda)}$ with some energy $E_n(\lambda)$, then
$$\frac{dE_n(\lambda)}{d \lambda}=\mel{\psi_n(\lambda)}{\frac{dH(\lambda)}{d \lambda}}{\psi_n(\lambda)}$$
\end{lemma}


\subsection{Numerical methods for the solution of the Schrodinger equation} \label{sec:numerical}
In the general case, the dynamics of a particle are described by a Hamiltonian: 
\begin{equation}\label{hamiltonian}
	\hat{H} =- \dfrac{\hbar}{\ {2}{m}} \nabla^2  + V(x),
\end{equation}
where $\hbar$ - Planck constant, ${m}$ - mass of particle, $\nabla^2 = \frac{\partial^2}{\partial x^2}$ - Laplasian,  $V$ - potential .\\
Schrodinger equation describes particle dynamics:
\begin{equation}\label{schr_eq}
	i \hbar \frac{\partial \Psi(x, t)}{\partial t} = H \Psi(x,t),
\end{equation}
where $\Psi(x, t)$ is a wave function, $H$ is a Hamiltonian.\\
Bohr provided statistical interpretation of the wave function: $|\Psi(x,t)|^2 $gives the probability of finding the particle at point $x$, at time $t$. Then,$|\Psi(x,t)|^2 dx $ gives probability of finding the particle at point $(x+dx)$, at time $t.$\\
It follows that 
$$\int_{-\infty}^\infty |\Psi(x,t)|^2 dx = 1$$
Such function is called normalized wave function.\\
Suppose we are finding solutions in the following form: $\Psi(x, t) = \psi(x) f(t)$
Time independent Schrodinger equation:
\begin{equation}\label{t-indep}
	H \psi = E \psi
\end{equation}
Notice, that general solution is a linear combination of the separable solutions: $\Psi(x,t) = \sum_{n=1}^\infty c_n \psi_n (x) e^{-i E_nt/ \hbar},$ where $\psi(x) = c e^{-iEt/ \hbar}.$
The problem is to find a wave function, describing the dynamic of the particle. As it turns out, it is often impossible to solve a differential equation analytically. In this case it is used numerical methods: describe the main idea.\\
In the spherical summetry case we obtain the following equation, which is called radial (see section \ref{e_sph_poten}): 
$$-\frac{\hbar^2}{2m}\frac{d^2 u}{d r^2}+[V+\frac{\hbar^2}{2m}\frac{l(l+1)}{r^2}]u = Eu,$$
where $u(r) = r R(r).$ 
If  $l=0$ the equation says:
$$-\frac{\hbar^2}{2m}\frac{d^2 u}{d r^2}+Vu = Eu,$$
$$-\frac{\hbar^2}{2m}u''+Vu = Eu.$$
If we let $\hbar = 1, m=1:$ (atomic units)
$$-\frac{u''}{2}+Vu = Eu,$$
$$u'' +(-2V)u = (-2E)u,$$
$$(\frac{d^2}{d r^2}-2V)u = (-2E)u.$$
The last equation explicitly hints at the search for eigenvalues of $(\frac{d^2}{d r^2}-2V)$. To do this, we need to approximate the continuous expression discrete. Since the wave function is typically exponentially damped, we can assume that $ u (r) = 0 $ for $ r \geq L $ for some sufficiently large $ L $. \\
Thus, we obtain an eigenvalue problem with the boundary conditions $u(0) = 0$(since $ u (r) = r R (r) $) and $ u (L) = 0 $. We define the step $ h = L N ^{-l}$, where $ N $ is a natural (divide the segment $ [0, L] $ into N parts, h is the step), $ r_n = nh $ is the grid nodes.\\
The second derivative is given by the following relation (from the Taylor series): $$\displaystyle{{u_n''} = \frac{u_{n+1}-2u_n +u_{n-1}}{h^2}}.$$ 
Then $(u''(r))_n \to u_n''(r):$
$$\displaystyle{u_n'' = -\frac{1}{h^2} \begin{bmatrix}
-2 & 1 &0&0&0& \cdots & 0 \\
1 & -2 & 1& 0&0&\cdots & 0 \\  
0&1 & -2 & 1& 0&\cdots & 0 \\  
\vdots & \vdots &\vdots & \vdots & \ddots & -2& 1  \\
0 & 0&0 &0& \cdots &1 & -2
\end{bmatrix}
\begin{bmatrix}
u_1
\\
u_2
\\
u_3
\\
\vdots
\\
u_n
\end{bmatrix}}$$
We note that the resulting matrix has a tridiagonal form.\\
Now we need approximate $(-2Vu)(r_n)$:
$$(-2Vu)(r_n) = \begin{bmatrix}
-2V(r_1) & 0 &0&0&0& \cdots & 0 \\
0 & -2V(r_2) & 0& 0&0&\cdots & 0 \\  
0&0 & -2V(r_3) & 0& 0&\cdots & 0 \\  
\vdots & \vdots &\vdots & \vdots & \ddots & \vdots& \vdots  \\
0 & 0&0 &0& \cdots &0 & -2V(r_{n-1})
\end{bmatrix}
\begin{bmatrix}
u_1
\\
u_2
\\
u_3
\\
\vdots
\\
u_n
\end{bmatrix}$$
It is easy to check that $\displaystyle{||u_n''-u''|| \sim O(h^2)}.$\\
Thus our problem is to find explicitly the eigenvalues for the sum of the matrices presented.


\section{Electron in spherically symmetric potentials}\label{e_sph_poten}
In this section we will try solve some problems with concrete potentials. But, firstly, we are going to consider some principles for solving  Schrodinger equation in three dimensions.\\
Using results from paragraph \ref{dif_op_3d} and Laplacian in spherical coordinates \eqref{sph_lap},  we find time-independent Schrodinger equation in the form:
\begin{equation}\label{t-indep_3d}
-\frac{\hbar^2}{2m}[\frac{1}{r^2} \frac{\partial}{\partial r}({r^2}\frac{\partial \psi}{\partial r})+\frac{1}{r^2\sin{\theta}}\frac{\partial}{\partial \theta}(\sin{\theta}\frac{\partial \psi}{\partial \theta})+\frac{1}{r^2\sin^2{\theta}}(\frac{\partial^2 \psi}{\partial \phi^2}) ] + V\psi = E \psi
\end{equation}
We are looking for solutions that are separable into products:
$$\psi (r, \theta, \psi) = R(r)Y(\theta, \psi),$$
putting this in \eqref{t-indep_3d} and using results from previous sections, we get:
\begin{equation}\label{r_eq}
	\frac{1}{R}\frac{\partial}{\partial r}(r^2 \frac{\partial R}{\partial r}) - \frac{2m r^2}{\hbar^2}[V(r)-E] = l(l+1),
\end{equation}
\begin{equation}\label{Y_eq}
	\frac{1}{Y}\frac{1}{\sin \theta}\frac{\partial}{\partial \theta}(\sin \theta \frac{\partial Y}{\partial \theta}) + \frac{1}{\sin^2 \theta}\frac{\partial^2 Y}{\partial \phi^2}=-l(l+1).
\end{equation}
The equation \eqref{Y_eq} is the angular part of wave function, \eqref{r_eq} is radial part of wave function. Notice, that  the angular part is the same for all spherically symmetric potentials, but for our purposes it is important to investigate the radial part, because spherically symmetric potent%\texttt{•}ials depend only on $r.$\\
Let $u(r) = r R(r)$, so:
\begin{equation}\label{r_eq}
	-\frac{\hbar^2}{2m}\frac{\partial^2 u}{\partial r^2}+[V+\frac{\hbar^2}{2m}\frac{l(l+1)}{r^2}]u = Eu
\end{equation}
It is so called radial equation. Normalization condition becomes
$$\int_{0}^{\infty} |u|^2  dr=1$$

\subsection{Spherically symmetric potential well}

\subsubsection{Infinite spherical well}
For infinite spherical well with radius $a$ potential is
$$V(r)= \begin{cases}
0, \text{if } r\leq a\\
\infty, \text{if } r>a
\end{cases}$$
Then, inside the well the radial equation \eqref{r_eq} says
\begin{equation}\label{rad_well}
\frac{d^2 u}{dr^2} = [\frac{l(l+1)}{r^2} - k^2]u,
\end{equation}
where
$$k = \frac{\sqrt{2mE}}{\hbar}$$
We need to solve this equation, using the boundary condition $u(a) = 0.$
Suppose, $l=0$:
$$\frac{d^2 u}{dr^2} = -k^2 u \Rightarrow u(r) = A \sin{(kr})+B\cos{(kr})$$
This case is easy. But notice, the actual radial wave function is $R(r)= u(r)/r$, and $(\cos (kr))/r$ blows up as  $r \to 0.$ Thus, $B =0.$ Then it is requires $\sin (ka) =0$, and hence $ka = n \pi,$ for some integer $n$. The allowed enerdies are
$$E_{n0} = \frac{n^2 \pi^2 \hbar^2}{2 ma^2}, ~ n = 1,2,...$$
To solve the equation in all cases define a new variable $z= k r$, and let  $u = z^{1/2}\phi{z}.$
Then \eqref{rad_well} takes the following form:
$$\frac{d^2 \phi}{dz^2}+\frac{1}{z}\frac{d \phi}{dz}+[1 - \frac{(l+\frac{1}{2})^2}{z^2}]\phi = 0.$$
This is Bessel's differential equation \eqref{bess_eq}. So, general solution is:
$$u(r)=Ar j_l(kr)+Bn_l(kr),$$
where $j_l(x)$ and  $n_l(x)$ are spherical Bessel functions \eqref{bess_func1}, \eqref{bess_func1} of order $l$.
Notice, that  \eqref{bess_func1} blow up at the origin. So we must have $B_l=0$, and hence
$$R(r)=A j_l(kr).$$
The boundary condition says $R(a)=0,$ which implies $k$ must be chosen such that
$$j_l(kr)=0.$$
This means, that $ka$ is zero of the $i^{th}$-order spherical Bessel function, but  they are not located in good points, such as $n$ or $\pi n$ . The boundary condition requires 
$$k = \frac{1}{a}\beta_{nl},$$
where $\beta_{nl}$ is the $n^{th}$ zero of the  $l^{th}$  spherical Bessel function. The allowed energies are defines by
\begin{equation}\label{sol_inf_well}
E_{nl}=\frac{\hbar^2}{2ma^2}\beta_{nl}^2,
\end{equation}
and the wave functions are
$$\psi_{nlm}(r, \theta,\phi) = A_{nl} j_l(\beta_{nl}r/a) Y_l^m(\theta, \phi),$$
where $A_{nl}$ is a constant determined by normalization, $Y_l^m(\theta, \phi)$ is spherical harmonics \eqref{sph_garm}.
We assume atomic units, so $\hbar = 1, m=1.$ As an example we consider finite spherical well with $\ a = 5 $ a.u. Results are presented in the following table: 


\begin{figure}[h!]
\centering
\begin{tabular}{lrrrr}
\toprule
{} &     $n=1$ &     $n=2$ &     $n=3$ &     $n=4$ \\
\midrule
$l=0$ &  0.197443 &  0.789522 &  1.776613 &  3.158087 \\
$l=1$ &  0.403741 &  1.193512 &  2.377944 &  3.957047 \\
$l=2$&  0.664243 &  1.654381 &  3.037127 &  4.814305 \\
$l=3$ &  0.976643 &  2.170278 &  3.752704 &  5.728436 \\
$l=4$ &  1.339230 &  2.740140 &  4.524032 &  6.698532 \\
\bottomrule
\end{tabular}
\caption{$E_{nl}$ in a.u. of an electron in an infinite spherical well of radius $a=5$ a.u.}
\end{figure}

 


\subsubsection{Finite spherical well}
For finite spherical well  potential is defined as:
$$V(r)=\begin{cases}
-V_0, ~\text{if } r< R\\
~~~~0, ~\text{if } r>R
\end{cases}$$
Let
$$\frac{2m|E|}{\hbar^2} = \varkappa^2, ~~~~\frac{2m V_0}{\hbar^2}=k_0^2, ~~~~\frac{2m(V_0-|E|)}{\hbar^2} =k^2,$$
then the radial equation inside of the well and outside are correspondently:
\begin{equation}\label{rad1}
	\frac{d^2 u}{du^2}+[k^2-\frac{l(l+1)}{r^2}]u=0
\end{equation}
\begin{equation}\label{rad2}
	\frac{d^2 u}{du^2}+[-\varkappa^2-\frac{l(l+1)}{r^2}]u=0
\end{equation}
The solution of the quation \eqref{rad1} is a spherical Bessel function (with the boundary confition $u(0) =0$):
$$r u(r) =A j_l(kr), ~r< R.$$
To find the solutions of \eqref{rad2} we need to define spherical Hankel functions:
$$h_l(x) = j_l(x)+in_l(x),$$
$$h_l^*(x)=j_l(x)-in_l(x),$$
where $j_l(x)$ and $n_l(x)$ are spherical Bessel functions \eqref{bess_func1}, \eqref{bess_func2}.\\
As $r \to \infty$
$$h_l(x) \to \frac{e^{[ix-\pi (l+1)/2]}}{x}$$
$$h_l^*(x) \to \frac{e^{-[ix-\pi (l+1)/2]}}{x}$$
The solution of the quation \eqref{rad2} is a spherical Hankel function of an imaginary argument:
$$r u(r) = B h_l^{(1)}(i \varkappa r), ~r>R.$$
We are interested in finding constants $A$ and $B$ from continuity and normalization condition. First match the wavefunction:
\begin{equation}\label{cond_1}
	A j_l(kR)=B h_l(i \varkappa R).
\end{equation}
Second match the first derivative:
\begin{equation}\label{cond_2}
	A k \bigg[\frac{d j_l(\rho)}{d\rho}\bigg]_{\rho = kR} = A i \varkappa \bigg[\frac{d j_l(\rho)}{d\rho}\bigg]_{\rho = i \varkappa R}.
\end{equation}
Now divide \eqref{cond_2} by \eqref{cond_1}, then the constants cancel out. This trick is called logarithmic derivative:
$$k\bigg[ \frac{\frac{d j_l(\rho)}{d \rho}}{j_l(\rho)} \bigg]_{\rho = kR} = i\varkappa \bigg[ \frac{\frac{d h_l(\rho)}{d \rho}}{h_l(\rho)} \bigg]_{\rho = i \varkappa R} $$
 Then, the solution of the problem depends on the value of $l$ . Let $l=0$, the boundary condition becomes
$$k \bigg[ \frac{\frac{\cos \rho}{\rho} - \frac{\sin \rho}{\rho^2}}{\frac{\sin \rho}{\rho}} \bigg]_{\rho = kR} = i \varkappa \bigg[ \frac{\frac{i e^{i \rho}}{i \rho} - \frac{e^{i \rho}}{i \rho^2}}{\frac{e^{i \rho}}{i \rho}} \bigg]_{\rho = i \varkappa R},
$$
divide and substitute for  $\rho$:
$$k \bigg(\ctg(kR)-\frac{1}{kR}\bigg) = i \varkappa \bigg( i - \frac{1}{i \varkappa R} \bigg),$$
\begin{equation}\label{eq_sol}
	k \ctg(kR) = - \varkappa.
\end{equation}
This is transcendental equation, but we can solve it numerically for  $k$,  for which $k_0^2-\varkappa^2=k^2.$ Also, it is possible that there can be no solution. For instance, we can solve this equation graphically, but remember, it is only for case $l=0.$ We are interested in more general cases.\\
For finding numerical solutions of our problem (our problem is to calculate the possible values of the electron energy) not only for $l=0$ it is convenient to use the following form of \eqref{eq_sol} with the notations $kR = x, ~k_0 R=x_0, ~\frac{k}{k_0} = \xi, ~\varkappa R = \sqrt{1=\xi^2}$:
\begin{equation}\label{eq_sol1}
	\tg(x_0 \xi) = f_l(x_0 \xi),
\end{equation}
where
$$f_0(x_0, \xi) = -\frac{\xi}{\sqrt{1-\xi^2}},$$
$$f_1(x_0, \xi) = \frac{x_0 \xi}{1+\frac{\xi^2}{1-\xi^2}(1+x_0 \sqrt{1-\xi^2})}.$$ 
It can be easily seen that in case $l=0$ \eqref{eq_sol1} takes the form \eqref{eq_sol}. In the further analysis we assume atomic units, so $\hbar = 1, m=1.$ By substituting values of $k_0, k$ and $\varkappa$ to the \eqref{eq_sol} we obtain:
$$\tan(\sqrt{2} a \sqrt{V_0-E})=-\sqrt{\frac{|E|}{V_0}-1}$$ 
As an example we consider finite spherical well with $\ a = 5 $ a.u.,$ V_0 = 10$ a.u. We solved the equation \eqref{eq_sol} using Newton's method. The resulting energies are given in the Figure~\ref{fig:finite_well_sol}.
\begin{figure}[h!]
\centering
\begin{subfigure}{.5\textwidth}
  \centering
  \includegraphics[width=1.0\linewidth]{tan-sqrt.pdf}
  \caption{$\tan(\sqrt{2} a \sqrt{V_0-E})=-\sqrt{\frac{|E|}{V_0}-1}$}
  \label{fig1:tan-sqrt}
\end{subfigure}%
\begin{subfigure}{.5\textwidth}
  \centering
  \includegraphics[width=1.0\linewidth]{ens_finite.pdf}
  \caption{Energy levels (a.u.)}
  \label{fig:finite_well_sol}
\end{subfigure}
\caption{Finite spherical well $V_0 = 10$ a.u., $a=5.0$ a.u.}
\label{fig:fin_well}
\end{figure}




\subsection{Isotropic Harmonic Oscillator}
For isotropic harmonic oscillator potential is defined as:
$$V(r) = \frac{1}{2} m \omega^2 r^2 $$
Isotropy means that the frequencies for independent $x$,$y$, $z$ oscillations are the same. So for the oscillator in 3 dimensions we can use creation and annihilation operators $\hat{a}_x^\dag$, $\hat{a}_y^\dag$ ,$\hat{a}_z^\dag$ and $\hat{a}_x$,  $\hat{a}_y$, $\hat{a}_z$ associated with 1D oscillators in the $x$,$y$,and $z$ directions. The Hamiltonian then takes the form:
$$H = \hbar \omega (\hat{N}_1 + \hat{N}_2 + \hat{N}_3 + \frac{3}{2}) = \hbar \omega (\hat{N} + \frac{3}{2}),$$
where $\hat{N} = \hat{N}_1 + \hat{N}_2 + \hat{N}_3 .$\\
The spectrum for this quantum mechanical system has degeneracies, that are explained by the existence of some hidden symmetry. Thus in some ways this quantum 3D oscillator is a lot more symmetric than the infinite spherical well.\\
For analytical solution we use the standard separation variables procedure, then the radial equation  $\eqref{r_eq}$ takes the form:
$$-\frac{\hbar^2}{2m}\frac{\partial^2 u}{\partial r^2}+[\frac{1}{2} m \omega^2 r^2+\frac{\hbar^2}{2m}\frac{l(l+1)}{r^2}]u = Eu$$
However, a  3D oscillator spectrum finding procedure is complicated, due to this reason we used the numerical methods, described in subsection  $\ref{sec:numerical}$, for solving the problem and compared it with the known formula:
\begin{equation}
	E_{nl} = \hbar \omega (2 n_r +l+\frac{3}{2}),~~ n_r = 0,~1,~2,...
\end{equation}
Results are the following:

\begin{figure}[h!]
\centering
\begin{subfigure}{.45\textwidth}
\begin{tabular}{lrrrr}
\toprule
{$n$} &     {Numerical} &     {Analytical}   \\
\midrule
$ 0 $&   0.5196\textcolor{red}{08} &    0.519615 \\
$ 1 $ &   1.212\textcolor{red}{398} &    1.212436 \\
$ 2$ &   1.905\textcolor{red}{165} &    1.905256 \\
$ 3 $ &   2.59\textcolor{red}{7907} &    2.598076 \\
$ 4$ &   3.290\textcolor{red}{626} &    3.290897 \\
$ 5 $ &   3.983\textcolor{red}{320} &    3.983717 \\
\bottomrule
\end{tabular}
\caption{$l=0$}
\end{subfigure}%
\begin{subfigure}{.45\textwidth}
\begin{tabular}{lrr}
\toprule
{$n$} &  Numerical &  Analytical \\
\midrule
0 &   0.8660\textcolor{red}{16} &    0.866025 \\
1 &   1.5588\textcolor{red}{07} &    1.558846 \\
2 &   2.251\textcolor{red}{574} &    2.251666 \\
3 &   2.944\textcolor{red}{318} &    2.944486 \\
4 &   3.637037 &    3.637307 \\
5 &   4.3\textcolor{red}{29732} &    4.330127 \\
\bottomrule
\end{tabular}
\caption{$l=1$}
\end{subfigure}
\caption{$E_{nl}$ in a.u. for harmonic oscillator }
\end{figure}
For more detailed information about numerical experiment see Appendix. For larger $l$ convergence is the same. \\
Illustration of energy levels given in  Figure \ref{fig:Harm_osc}:
\begin{figure}[h!]
\centering
\begin{subfigure}{.5\textwidth}
  \centering
  \includegraphics[width=1.0\linewidth]{harm_osc.pdf}
  \caption{Spectrum in case $l=0$ (a.u.)}
  \label{fig1:harm_osc}
\end{subfigure}%
\begin{subfigure}{.5\textwidth}
  \centering
  \includegraphics[width=1.0\linewidth]{harm_osc1.pdf}
  \caption{Spectrum in case $l=1$( a.u.)}
  \label{fig:harm_osc1}
\end{subfigure}
\caption{Isotropic Harmonic Oscillator}
\label{fig:Harm_osc}
\end{figure}


\subsection{The Hydrogen Atom}
From the Coulomb's law, the potential energy (in a.u.) is 
\begin{equation}\label{h2_potrntial}
	V(r) = -\frac{1}{r},
\end{equation}
and the radial equation $\eqref{r_eq}$  says
$$-\frac{1}{2}\frac{d^2 u}{dr^2}+[-\frac{1}{r}+\frac{1}{2}\frac{l(l+1)}{r^2}]u = Eu.$$
The analytical solution is known:
\begin{equation}\label{h2_sol}
   E_n = -\frac{1}{2 n^2}
\end{equation}
Notice, that the hydrogen's spectrum depends only on the energy level $n.$ The reason that the energy level depends only on n is that for just one electron the only force acting on it is the coulomb force between the nucleus and the electron. As such this is the only force that dictates the potential energy term in the Schrodinger equation. In this case the strength of the coulomb force between the hydrogen nucleus and the electron only depends on the distance between the two and the charge involved (2q in this case). The force is spherically symmetric (aka rotationally invariant) so it doesn't matter where around the hydrogen nucleus an electron is. As long as the distance between the two is constant the strength of the force is constant. Hence, the energy is constant.\\
We used the numerical methods from section $\ref{sec:numerical},$ and results are the following:
\begin{figure}[h!]
\centering
\begin{tabular}{lrr}
\toprule
\centering
$n$ &         Numerical &         Analytical \\
\midrule
1 & -0.\textcolor{red}{499911} & -0.500000 \\
2 & -0.12\textcolor{red}{4994} & -0.125000 \\
3 & -0.05555\textcolor{red}{4} & -0.055556 \\
4 & -0.031250 & -0.031250 \\
5 & -0.020000 & -0.020000 \\
\bottomrule
\end{tabular}
\caption{$E_{nl}$ in a.u. of The Hydrogen Atom}
\end{figure}
We have some problems with the convergence, because the potential $\eqref{h2_potrntial}$ is singular at point 0. Graphical illustration of the spectrum is the followng:
\begin{figure}[h!]
\centering
\begin{subfigure}{.2\textwidth}
  \centering
  \includegraphics[width=1.0\linewidth]{h2_atom.pdf}
\end{subfigure}
\caption{The Hydrogen Atom Spectrum (in a.u.)}
\end{figure}

\subsection{Woods-Saxon potential}
The Woods-Saxon potential describes the interaction of the protons and neutrons with the heavy nucleus inside the atomic nucleus:
$$V(r) = -\frac{V_0}{1+e^\frac{r-R}{a}},$$
where $a<<R.$\\ 
$R$ is a the nuclear radius, $a$ is the parameter that characterizes the thickness of a surface layer within which the potential falls from $V=0$ outside the nuclear to  $V = -V_0$ inside one. If let $a=0$, we obtain a potential well with a potential jump at the nuclear surface. \\
The radial equation $\eqref{r_eq}$  says:
$$-\frac{\hbar^2}{2m}\frac{\partial^2 u}{\partial r^2}+[-\frac{V_0}{1+e^\frac{r-R}{a}}+\frac{\hbar^2}{2m}\frac{l(l+1)}{r^2}]u = Eu.$$
When using the Schrodinger equation to find the spectrum of nucleons subjected to the Woods–Saxon potential, it is impossible to find a analytical solution. Numerical method results are the following:
\begin{figure}[h!]
\centering
\begin{subfigure}{.5\textwidth}
  \centering
  \includegraphics[width=1.0\linewidth]{WS_poten.pdf}
  \caption{Spectrum in case $l=0$ (a.u.)}
  \label{fig1:WS_spc}
\end{subfigure}%
\begin{subfigure}{.5\textwidth}
  \centering
  \includegraphics[width=1.0\linewidth]{WS_wave.pdf}
  \caption{Wave functions in case $l=1$}
  \label{fig:wave_func}
\end{subfigure}
\caption{Woods-Saxon potential}
\label{fig:WS_potential}
\end{figure}

It is also possible  to solve this problem using  the Nikiforov-Uvarov method. 


\subsection{Hulthen Potential}
The Hulthen potential is given by
\begin{equation}
V(r) = - V_0 \frac{e^{-\frac{r}{a}}}{1 - e^{-\frac{r}{a}}}.
\end{equation}
Our problem to find the spectrum in case $l=0.$\\
The radial part $\eqref{r_eq}$ of the Schrodinger equation in a.u.  says:
$$-\frac{1}{2}\frac{\partial^2 u}{\partial r^2}+[- V_0 \frac{e^{-\frac{r}{a}}}{1 - e^{-\frac{r}{a}}} +\frac{1}{2}\frac{l(l+1)}{r^2}]u = Eu.$$
The analytical solution is known:
\begin{equation}\label{hult_sol}
E_n = -V_0\bigg(\frac{\beta^2 - n^2}{2n\beta}\bigg)^2,
\end{equation}
\begin{equation}\label{hult_cond}
\beta^2 > n^2, 
\end{equation}
where $\beta^2 = \displaystyle{\frac{2m V_0}{\hbar^2}a^2} > 0.$\\
Rewrite \eqref{hult_sol} in the atomic units:
$$E_n = -V_0\bigg(\frac{2 V_0 a^2 - n^2}{2n a \sqrt{2V_0}}\bigg)^2,$$
and compare it with the numerical methods from section  \ref{sec:numerical}:
\begin{figure}
	\centering
	\begin{tabular}{lrr}
		\toprule
		\centering
		$n$ &         Numerical &         Analytical \\
		\midrule
		1 & -\textcolor{red}{7.99988128221} & -8.0 \\
		2 & -0.12\textcolor{red}{4924678667} & -0.125 \\
		\bottomrule
	\end{tabular}
	\caption{$E_{n0}$ in a.u. for the Hulthen Potential}
\end{figure}\\
In our calculation we let $V_0 = 10 a.u.,~a = 0.5,$ so from the condition \ref{hult_cond}: $2.5 > n^2,$ we obtain only two energy levels.\\
Notice, that the Hulthen potential behaves like  a Colomb potential at small values of $r:$
$$V_c = -\frac{V_0 a}{r},$$
and for large values of $r$ it decreases exponentially so that its "capacity" for bound states is smaller than that of $V_c.$


\subsection{Morse potential}
The Morse potential describes the vibrations of a two-atomic molecule:
\begin{equation}
V(r) = D(e^{-2\alpha x} - 2e^{-\alpha x}), 
\end{equation}
$$x= \frac{r-r_0}{r_0}.$$
This formula is used to model the atom-surface interaction, and  $r$ is the coordinate perpendicular to the surface, $r_0$ is the equilibrium bond distance, $D$ is the well depth and $\alpha$ controls the 'width' of the potential (the smaller $\alpha$ is, the larger the well). \\
The parameters $D$, $\alpha$ and $r_0$ are known for some typical molecules such as $H_2$ or $H Cl$ from experiments. We will research the hydrogen molecule with the following parameters: $D = 38292 \quad cm^{-1}$, $\alpha= 1440$, $r_0 = 0.13984$ and mass equal to $M = 2(m_p + m_e)$ where $m_p$ and $m_e$ are the masses of proton and electron correspondingly. The problem is to find the spectrum of bound states for $l=0$. \\
The radial equation 
in a. u.  says:
$$-\frac{1}{2m}\frac{\partial^2 u}{\partial r^2}+[D(e^{-2\alpha x} - 2e^{-\alpha x})+\frac{1}{2m}]u = Eu.$$
The analytical solution is known:
$$E_n = -D + \hbar \omega [(n+\frac{1}{2}) - \frac{1}{\xi}(n+\frac{1}{2})^2],$$
where $\xi = \frac{2 \gamma}{\alpha},$ $\gamma^2 = \frac{2 M D r_0^2}{\hbar^2},$ and $\omega$ can be found from the following equation: $\hbar \omega = \frac{\hbar^2}{2 M  r_0^2}.$\\
We used the numerical method from section \ref{sec:numerical}. Results are the following:

\begin{figure}[h!]
	\centering
\begin{tabular}{lrr}
	\toprule
	n & Numerical & Analytical \\
	\midrule
	0 & -0.70\textcolor{red}{6884} & -0.707390 \\
	1 & -0.68\textcolor{red}{6828} & -0.688970 \\
	2 & -0.6\textcolor{red}{67061} & -0.671669 \\
	3 & -0.6\textcolor{red}{47583} & -0.655489 \\
	4 & -0.6\textcolor{red}{28393} & -0.640428 \\
	\bottomrule
\end{tabular}
	\caption{$E_{n0}$ in a.u. for the Morse potential}
\end{figure}

Notice, that at large distances the potential corresponds to the attraction forces, it comes to a minimum $-D$ at $x=0$ or $r=r_0$, but produces a strong repulsion if the two nuclei approach even closer.\\
Around $x=0$ it may be expanded into a series:
\begin{equation}\label{morse_seq}
V(r) = D(-1 + \alpha^2 x^2+...) = -D + \frac{1}{2}M \omega^2(r-r_0)^2 +...,
\end{equation}
where $\omega^2 = \frac{2D \alpha^2}{M r_0^2}.$\\
Thus, for low vibrational levels, a spectrum is not
deviating very much from that of a harmonic oscillator:
$$E (\nu) = -D + \hbar \omega (\nu + \frac{1}{2}),~~\nu = 0,~1,~2, ...,$$
with $\nu$ the vibrational quantum number Notice that almost equidistant terms become denser when energy increases, in consequence of the anharmonicity  in \ref{morse_seq}. 
\begin{figure}[h!]
	\centering
	\begin{subfigure}{.5\textwidth}
		\centering
		\includegraphics[width=1.0\linewidth]{morse_pot.pdf}
	\end{subfigure}
	\caption{Morse potential}
\end{figure}

\subsection{Yukawa Potential}
The Yukawa potential is used to  describe the nuclear attraction between two particles (for example, nucleons), it is often used to compute the spectrum of neutral atoms.\\ 
The Yukawa potential is given by
\begin{equation} \label{yukawa_pot}
V(r) = -V_0\frac{e^{a r}}{r},
\end{equation}
where $V_0 = \alpha Z$ and $Z$ is the atomic number, $\alpha = \frac{1}{137}$ is the fine-structure constant and $a$ is the screening parameter.
Our problem is to find the spectrum. The radial part $\eqref{r_eq}$ of the Schrodinger equation in a.u. becomes:
$$-\frac{1}{2}\frac{\partial^2 u}{\partial r^2}+[-V_0\frac{e^{a r}}{r} +\frac{1}{2}\frac{l(l+1)}{r^2}]u = Eu.$$
No exact analytical solution was found, however some approximate formula for energy levels was obtained using Nikiforof-Uvarov method: 
\begin{equation}\label{yukawa_sol}
E_{nl} = -\frac{a^2}{2m}\frac{(\displaystyle{\frac{m V_0}{a}} - (n+1)^2 - l(2n+l+2))^2}{(n+l+1)^2}
\end{equation}
Let us compare results of the numerical method from the section \ref{sec:numerical} and the formula \eqref{yukawa_sol} if $l = 0$:
\begin{figure}[h!]
	\centering
	\begin{tabular}{lrr}
		\toprule
		{n} &  Numerical   &         App. Analytical \\
		\midrule
		0 & -0.09\textcolor{red}{6186} & -0.097882 \\
		1 & -0.024\textcolor{red}{362} & -0.024470 \\
		2 & -0.0108\textcolor{red}{54} & -0.010876 \\
		3 & -0.00611\textcolor{red}{1} & -0.006118 \\
		4 & -0.00391\textcolor{red}{2} & -0.003915 \\
		5 & -0.00271\textcolor{red}{7} & -0.002719 \\
		6 & -0.001997 & -0.001997 \\
		7 & -0.001529 & -0.001529 \\
		8 & -0.001208 & -0.001208 \\
		9 & -0.00097\textcolor{red}{8} & -0.000979 \\
		\bottomrule
	\end{tabular}
	\caption{$E_{n0}$ in a.u. for the Yukawa Potential}
\end{figure}\\
If the screening parameter $ a\to 0$, the potential \eqref{yukawa_pot} becomes the Coulomb potential. Hence, the eigenvalues of \eqref{yukawa_sol} become the spectrum of pure Coulomb potential:
$$E_{nl}^{Coulomb} = -\frac{1}{2} m \frac{V_0^2}{n_l^2},$$
where $n_l = n+l+1.$

\subsection{Central-force model of deuteron}

The neutron-proton interaction in the idealized situation is described by the following potential:
$$V(r) = - A e^{-r/a}$$
Our problem is to solve the Schrodinger equation for the bound state with $l=0$ (deuteron). As it was determined by experiment there exists only one bound state:
\begin{equation}
    E = -2.23~\mathrm{MeV}
\end{equation}


\subsubsection{Exact solution}
For $l = 0$ we obtain the following radial Schrodinger equation \eqref{r_eq}:
\begin{equation}
    \frac{d^2 u}{d r^2} + \frac{2 m^*}{\hbar^2}(E+ A e^{-r/a})u = 0,
\end{equation}
where $m^* = \frac{1}{2}$, because we let masses of both nucleons equal. Use of the new variable:
\begin{equation}\label{abbr}
    y = e^{-r/2a}
\end{equation}
we obtain:
\begin{equation}\label{eq_bess_d}
    \frac{d^2 u}{d y^2} + \frac{1}{y}\frac{d u}{d y} + (c^2 - \frac{q^2}{y^2})u=0,
\end{equation}
where
\begin{equation}\label{abbr_2}
    c^2 = \frac{8 m^*}{\hbar^2}A a^2,~~q^2 = -\frac{8 m^*}{\hbar^2}E a^2 > 0
\end{equation}
The equation  \eqref{eq_bess_d} is Bessel's differential equation \eqref{bess_eq} with the general solution:
$$u = C_1 J_q(cy) + C_2 J_{-q} (cy).$$
One the one hand, $r$ blows up if $y = 0$ where $u$ must vanish. Therefore $C_2 = 0:$
\begin{equation}\label{ex_sol_deuteron}
    u = \frac{C_1}{r}J_q(c e^{-r/2a})
\end{equation}
On the other hand, $y = 1$ corresponds to $r = 0$ where $u$ must be finite so that the Bessel function in bound to vanish:
$$J_q(c) = 0.$$
On the grounds of known numerical values of $m^*$ and $E$ we obtain from \eqref{abbr_2}:

\subsubsection{Approximate solution}
The wave function
$$\widetilde{u} = \frac{C}{r}\displaystyle{\big(e^{-\frac{qr}{2a}} - e^{-\frac{qr}{a}}\big)} $$
has a finite value
$$u(0) = C\frac{q}{2a}$$
at $r=0,$ and decreases, for large $r,$ exponentially with the correct decay length:
$$\exp\bigg(-\frac{qr}{2a}\bigg) = \exp\bigg(-r \sqrt{\frac{2 m^* |E|}{\hbar^2}}\bigg).$$
The normalization constant $C$ can be obtained from 
$$4 \pi \displaystyle{\int\limits_{0}^ {\infty}} dr~ r^2 \widetilde{u}(r)^2 = 1;$$
\begin{equation}
    C = \sqrt{\frac{3q}{2 \pi a}}.
\end{equation}
We determine the expectation value of the energy with this wave function by using the Schrodinger variational principle:
\begin{equation}\label{integral}
    E = 4 \pi \int\limits_{0}^{\infty} dr ~r^2 \bigg[\frac{\hbar^2}{2 m^*}(du/dr)^2 -A e ^{-r/a} u^2\bigg]
\end{equation}
The exact solution \eqref{ex_sol_deuteron} put into \eqref{integral}, the exact eigenvalue $E$ would result. Using instead the approximate function $\widetilde{u}$, we find the approximate eigenvalue $\widetilde{E}$.\\
The evaluation of \eqref{integral} with \eqref{ex_sol_deuteron} has the following form:
\begin{equation}
    E = \frac{\hbar^2}{4 m^* a^2}q^2 - 6 A \bigg[\frac{1}{1 + \frac{1}{q}} - \frac{1}{\frac{3}{2} + \frac{1}{q}} +  \frac{1}{2 + \frac{1}{q}} \bigg]
\end{equation}
Some numerical results: $\widetilde{E}$=-2.18 MeV if $a$ = 2.18 fm and $A$ = 32.7 MeV. The correct value $E$ = -2.23 MeV.


\section{Conclusion}
In this work I have researched the spectrum problem of a particle in some spherically symmetric potentials by using a numerical method called finite difference method. The numerical results are obtained by this method and compared with the analytical solution. 




\section{Appendix}\label{appendix}
The purpose of this section is to expand on some of the techniques and values used in numerical computations. All programs are written on Python 3.
\begin{center}
		\large
		\textbf{ Infinite spherical well}		\\[0.5 cm]
\end{center}
The potential is defined by 
$$E_{nl} = \frac{\hbar^2}{2ma^2}\beta_{nl}^2,$$
where $\beta_{nl}$ -  $n^{th}$ zero of the $l^{th}$ spherical Bessel function $j_n(x) = (-x)^n\bigg(\displaystyle{\frac{1}{x} \frac{d}{dx}}\bigg)^n \frac{\sin{x}}{x}.$
Zeros are known:
\begin{figure}[h!]
\centering
\begin{tabular}{lrrrr}
\toprule
{} &  \ $n=1$ &   $n=2$ &   $n=3$ &   $n=4$ \\
\midrule
$l=0$ &  3.142 &   6.283 &   9.425 &  12.566 \\
$l=1$ &  4.493 &   7.725 &  10.904 &  14.066 \\
$l=2$ &  5.763 &   9.095 &  12.323 &  15.515 \\
$l=3$ &  6.988 &  10.417 &  13.698 &  16.924 \\
$l=4$ &  8.183 &  11.705 &  15.040 &  18.301 \\
\bottomrule
\end{tabular}
\caption{Bessel Function Zeros}
\end{figure}
We assume atomic units, so $\hbar =  1$, $m=  1$ .As an example we consider infinite spherical well with $a= 5$ a.u.
\begin{lstlisting}[frame=single]
#Bessel function zeros array
n = 5
A = [] 
for i in range(n): 
    A.append(list(map(float, input().split())))
    
a=5
E = Arr**2 / (2.0 * a**2)
df = pd.DataFrame(E)
df.columns = [r"$n=1$", r"$n=2$", r"$n=3$", r"$n=4$"]
df
\end{lstlisting}
The result of running this code is represented on Figure \ref{fig:inf_well}.
\pagebreak


\begin{center}
		\large
		\textbf{Finite spherical well}		\\[0.5 cm]
\end{center}
In case $l=0$ the particle spectrum is defined by the following formula:



%$$\lambda \sin^2 {\theta}+\frac{\sin{\theta}}{\theta}\frac{\d}{\d \theta}$$

%\bibliography{electron}
%\bibliographystyle{plain}


%\section{Appendix}\label{appendix}

%$$\lambda \sin^2 {\theta}+\frac{\sin{\theta}}{\theta}\frac{\d}{\d \theta}$$

\bibliography{electron}
\bibliographystyle{plain}
\end{document}